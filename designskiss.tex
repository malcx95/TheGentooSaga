\documentclass[a4paper,titlepage]{article}
\usepackage[utf8]{inputenc} %Make sure all UTF8 characters work in the document
\usepackage{listings} %Add code sections
\usepackage{color}
\usepackage{graphicx}
\usepackage{titling}
\usepackage{textcomp}
\usepackage[hyphens]{url}
\usepackage[bottom]{footmisc}
\usepackage[yyyymmdd]{datetime}
\definecolor{listinggray}{gray}{0.9}
\definecolor{lbcolor}{rgb}{0.9,0.9,0.9}
\lstset{	%ifall du ska koda.
		backgroundcolor=\color{lbcolor},
		tabsize=4,
		rulecolor=,
		upquote=true,
		aboveskip={1.5\baselineskip},
		columns=fixed,
		showstringspaces=false,
		extendedchars=true,
	    breaklines=true,
	    prebreak = \raisebox{0ex}[0ex][0ex]{\ensuremath{\hookleftarrow}},
	    frame=single,
	    showtabs=false,
	    showspaces=false,
	    showstringspaces=false,
	    identifierstyle=\ttfamily,
	    keywordstyle=\color[rgb]{0,0,1},
	    commentstyle=\color[rgb]{0.133,0.545,0.133},
	    stringstyle=\color[rgb]{0.627,0.126,0.941},
}

%Set page size
\usepackage{geometry}
\geometry{margin=3cm}
\usepackage{parskip} 

\renewcommand{\dateseparator}{-}

\title{
\textbf{Kravspecifikation - The Gentoo Saga} \\
\large TSEA83 Grupp 33}
    \date{\today}
\author{
        Emil Segerbäck - emise935 - emise935@student.liu.se\\
		Malcolm Vigren - malvi108 - malvi108@student.liu.se \\
		Robin Sliwa - robsl733 - robsl733@student.liu.se }
\begin{document}
    \maketitle
    \newpage
\tableofcontents
    \newpage

%\section{Inledning}
%
%\newpage
%\section{Analys av problemet}
%
%\newpage
%\section{Förfinat blockschema}
%
%\newpage
%\section{Milstolpe}
 
\newpage
\section{Checklista}
    \subsection{Spelimplementation}
        \begin{itemize}
            \item Se bild på discord
            \item Bland annat att PS2-kontrollern skriver till ett register om
                vilka knappar som är nedtryckta, och endast de som används i
                spelet. Vi har också spriteminnen.
        \end{itemize}
    \subsection{CPU}
        \begin{itemize}
            \item Pipeline
            \item Separata program/dataminnen
            \item De flesta, om inte alla, instruktioner från lab 2 plus några
				fler.
            \item Processorn ska arbeta med 32-bitars ord.
            \item Eftersom vi använder pipeline är addresseringsmoder inte applicerbart.
            \item Specialinstruktioner för tiles, sprites och musik.
            \item Program- och dataminnet ska programmeras via UART.
            \item Vi ska använda ett separat hårdvarublock för detta.
            \item Processorn sköter all spelmekanik - AI,
				kollisionsdetektering, rörelseberäkning etc.
        \end{itemize}
	\subsection{Grafik}
        \begin{itemize}
			\item Grafiken ritas ut av en separat hårvarukomponent m.h.a. tile-
				och spriteminne.
			\item 320x240, 8-bitars färg. Tiles och sprites á 16x16 pixlar.
			\item CPUn skriver skiten i bildminnet och sedan ritar
				grafikenheten ut grejerna, så ingen synkronisering behövs.
        \end{itemize}
	\subsection{I/O-enheter}
        \begin{itemize}
			\item Tangentbord, högtalare och skärm.
			\item Tangentbordet skriver till ett register i CPUn.
        \end{itemize}
	\subsection{Minnesanvändning}
        \begin{itemize}
			\item Alla register är 32 bitar breda. Vi använder ett
				programminne, dataminne, tileminne, ett extra tileminne för
				bakgrund, spriteminne och musikminne.
			\item Tileminnet har 15x150 tiles. Vi har 32 olika tiles så
				tileminnet behöver 15x150x5=11250 bitar=1407 bytes. Det andra
				tileminnet behöver 15x75x4=4500 bitar=563 bytes. Spriteminne
				behöver bara innehålla två sprites på 16x16 pixlar. Musikminne
				behöver instruktioner som har 5 bitar för tonhöjd och 2 bitar
				för tonlängd, totalt 8x128=1024 bitar=128 bytes. Programminnet
				behöver 2000 instruktioner totalt 32x2048=8192 bytes. Till
				dataminnet tar vi 2048 bytes.
			\item CPUn ska kunna läsa från programminnet och läsa/skriva till
				dataminnet. Den ska även kunna starta en överföring från
				dataminne till tile-, sprite- och musikminnet. 
			\item Ja, vi uppskattar att minnet kommer att räcka.
        \end{itemize}
	\subsection{Programmering}
        \begin{itemize}
			\item En assembler ska skrivas. UART används för överföring. 
        \end{itemize}
	\subsection{Milstolpe}
        \begin{itemize}
			\item Processorn fungerar.
			\item Grafiken ska kunna rita ut tiles på VGA-skärmen.
			\item Man kan överföra data m.h.a. UART.
        \end{itemize}
	\subsection{Blockscheman}
        \begin{itemize}
			\item 
        \end{itemize}
\end{document}
